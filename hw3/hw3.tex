\documentclass{article}

\usepackage{fullpage} % Include this if you want to cram lots of things on a page
 
\usepackage{amsmath} % these are standard macro packages of the American Mathematical Society
\usepackage{amssymb}
\usepackage{hyperref}
%\usepackage{stmaryrd}

\usepackage{epsfig} % if you want figures

%\usepackage{fancyhdr} % These 4 lines are needed to set up the running  header
%\fancyhead[LE,RO]{Katherine Scott - Homework 1}
%\fancyhead[RE,LO]{\thepage}
%\pagestyle{fancy}

\newcommand{\matlab}[1]
{\centerline{\parbox{.9\textwidth}{\noindent\textsc{\bf MATLAB:} #1}}}

\newcommand{\code}[1]{\texttt{#1}}

\newcommand {\x}{\V{x}}
\newcommand {\y}{\V{y}}
\newcommand {\V}[1]{\mbox{\boldmath$#1$}}
% To add some paragraph space between lines.
% This also tells LaTeX to preferably break a page on one of these gaps
% if there is a needed pagebreak nearby.
\newcommand{\blankline}{\quad\pagebreak[2]}

\begin{document}
\title{Computer Vision: Homework 5}

\author{Katherine A. Scott}
\maketitle
\mbox{}
\begin{center}
\href{mailto:katherineAScott@gmail.com}{kas2221@columbia.edu}

\end{center}
\section{Problem 1}
The general formula for the irradiance at any point is:
\[
I=\frac{\rho}{\pi}kc\cos{\theta_i}=\frac{\rho}{\pi}kc\textbf{n}\cdot\textbf{s}
\]
We can assume, since we have constant brightness, and albedo at a
point that the $\frac{\rho}{\pi}kc$ is constant. We can also assume
that illumination is adaptive. Such that:
\[
I = I_1+I_2 = \frac{\rho}{\pi}kc(\textbf{n} \cdot \textbf{s}_1 + \textbf{n} \cdot
\textbf{s}_2) 
\]
If we multiply this all out we see that 
\[
I = \frac{\rho}{\pi}kc \textbf{n} \cdot (\textbf{s}_1+\textbf{s}_2)
\] 
And our light source acts just as the addition of the two
vectors. When the two vectors do not have the same brightness values,
$k$ of the $\frac{\rho}{\pi}kc$ terms are not equal. We get the
following 
\[
I=\frac{\rho}{\pi}c(k_1\textbf{n} \cdot \textbf{s}_1+
k_2\textbf{n} \cdot \textbf{s}_2)=\frac{\rho}{\pi}c\textbf{n}\cdot(k_1\textbf{s}_1+k_2\textbf{s}_2)
\]
We can see from this arrangement that the effective lighing direction
is just the addition of the two direction vectors scaled by the
relative brightness of the light source. 
\section{Problem 2}
\subsection{Part 1}
The curves of constant brightness are just circles inscribed on the
Gaussian sphere. This is because on a Lambertian surface the
irradiance is a function of the the surface normal and the lighting
direction. The surface of the sphere describes a surface normal, and
the light source solutions represent a cone. The intersection of the
cone and the sphere describe a circle.  
\subsection{Part 2}
Since the contours are just circles on the sphere, the circles can
only intersect at zero, one, or two points for a solution. Circles can only overlap
at most two places. 
\end{document}